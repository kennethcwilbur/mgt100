\documentclass[12pt]{article}

%%% PACKAGES
\usepackage[parfill]{parskip}       % begin paragraphs with an empty line rather than an indent
\usepackage{booktabs}               % for much better looking tables
\usepackage{paralist}               % very flexible & customisable lists (eg. enumerate/itemize, etc.)
\usepackage{verbatim}               % adds environment for commenting out blocks of text & for better verbatim
\usepackage{hyperref}               % for web links and email addresses
\usepackage{enumitem}               % for controlling spacing of items in lists
\usepackage[margin=1in]{geometry}   % for controlling page margins
\usepackage{color}
 \usepackage{setspace}

%%% HEADERS & FOOTERS
\usepackage{fancyhdr} % This should be set AFTER setting up the page geometry
\pagestyle{fancy} % options: empty , plain , fancy
\renewcommand{\headrulewidth}{0pt} % customise the layout...
\lhead{}\chead{}\rhead{}
\lfoot{}\cfoot{\thepage}\rfoot{}
%\doublespacing
%%% OTHER FORMATTING
\renewcommand\labelitemi{\tiny $\bullet$} % smaller bullets in 

%%% TITLE & AUTHOR
\title{\vspace{-1cm}Syllabus:\vspace{-0.3cm}}
\author{\vspace{-0.5cm}UCSD MGT 100, Customer Analytics}
%\date{\vspace{-0.2cm}Spring 2023}


% ----------------------------------------------------------------------------------------------------

\begin{document}
\maketitle



\section*{Instructors}
\medskip
\hrule
\medskip

%Class: MGT 100: Customer Analytics
%\vspace{-0.15cm}
%\begin{itemize}[noitemsep]
%\item canvas: \href{https://FILLINLINK.edu}{FILLINLINK}
%\item piazza: \href{https://FILLINLINK.edu}{FILLINLINK}
%\end{itemize}
%\medskip

Professor: Kenneth C. Wilbur \textcolor{blue}{\href{http://kennethcwilbur.com/}{Website} \href{https://github.com/kennethcwilbur/website/raw/master/kennethcwilbur_cv.pdf}{CV}}. \\Office hours: Mondays, 1:30-3 PM, Rady 3W117 or \textcolor{blue}{\href{https://ucsd.zoom.us/j/9477848814}{Zoom}}
\vspace{0.2cm}

Teaching Assistant: \textcolor{blue}{\href{https://seunghyun-kim.com/}{Seung Hyun Kim}}.\\
Teaching Assistant: \textcolor{blue}{\href{https://www.linkedin.com/in/gabriel-ca\%C3\%B1edo-riedel-612443147/}{Gabriel Canedo Riedel}}. 
\\Office hours: TBA
\vspace{0.2cm}

\section*{Welcome}
\medskip
\hrule
\medskip
We welcome everyone to this course. We want all students to feel valued and safe. 

We want you to succeed. We will work hard to help make that happen. You will need to participate actively, knowing that we have good intentions toward you, and investing the time and effort needed to learn this valuable class material.

This is not an easy class for most people. We cover challenging material quickly. Yet please rest assured that nearly all students pass this course. We have received overwhelmingly positive comments about the value of the course, the style in which we taught it, and the skills students gained as a result of studying this material.

We understand that student learning styles differ and no single approach is best for everyone. We also know that anyone can go through a difficult time. Please tell us if you have trouble learning in this environment. We may be able to make suggestions, connect you with resources, or find appropriate accommodations. We will work with you as best we can.

\vspace{0.2cm}


\section*{Navigation}
\medskip
\hrule
\medskip

% ----------
%\subsubsection*{Schedule}

%We meet WEEKDAY from TIME -- TIMEpm in Room PLACE of BUILDING.

%If we meet remotely, we will use this zoom link: \href{https://ucsd.zoom.us/j/9477848814}{link}.

%All lectures will be recorded and posted on Canvas, assuming no technical difficulties.


% ----------
%\subsubsection*{Materials}

The course outline, slides, readings, data, class scripts and class recordings are online at \\ \textcolor{blue}{\href{https://kennethcwilbur.github.io/mgt100/}{https://kennethcwilbur.github.io/mgt100/}}. 

%We rely solely on freely-available books and articles, including \href{https://r4ds.hadley.nz/}{\emph{R for Data Science}} by Wickham, Çetinkaya-Rundel and Grolemund and \href{https://eml.berkeley.edu/books/choice2.html}{\emph{Discrete Choice Methods with Simulation}} by Train.

Canvas is (only) for turning in deliverables, study groups and grades. 

Piazza is (only) for announcements, messages and discussions.

%\vspace{.5cm}

{\it Communication Policy:} 
\vspace{-0.15cm}
\begin{itemize}[noitemsep]
\item All student questions and comments should be posted on Piazza to ensure that all students can access all shareable information. 
\item Piazza messages can be private or public. Messages about class material and policies should be made public, but can be set as anonymous to peers if desired. Private messages are for personal issues such as family emergency or illness. 
\item We will check Piazza every weekday, before we check email and Canvas messages, so Piazza answers will be fastest. 
\item Kindly do not disregard our communication policy by sending emails or Canvas messages. If you do, we will kindly ask you to post your question on Piazza before we answer it. 
\item The best time to catch the professor for a chat is shortly after class or during the break. Please be judicious before class if he is still setting up. 
\item Office hours are for open-ended conversations that would be inefficient on Piazza. They occur on a drop-in basis: we meet students in the order they arrive, without offering or requiring appointments. 
\end{itemize}


\section*{Course Introduction}
% --------------------------------------------------

\medskip
\hrule
\medskip


Customer Analytics use of customer data to improve decisions. Usually, customer data are augmented by domain knowledge, reelvant theory, statistics and/or econometric modeling to inform and improve business policies.

MGT 100 was designed from scratch by UCSD faculty for quantitative UCSD students. It serves as a core course in the Business Economics major and an alternate core course in the Business and Marketing minors.

Our primary goal is to develop student understanding of data-driven business decision-making. Our secondary goal is to enable students to perform and interpret analytic techniques using code.

MGT 100 is designed as a survey course: We cover a broad range of topics in limited depth. We also have a deeper through-line that investigates demand modeling and usage. The survey nature of the course is more typical of graduate business classes than the economics classes that many students will have taken previously.

%We focus more on timeless principles than current issues. We focus on a somewhat idealized set of analytics frameworks in which firms estimate heterogeneous demand models and use those models to evaluate counterfactual policies. These frameworks are used in many sophisticated organizations, but not all firms use them, as they require skilled personnel, sophisticated understanding, proper incentives and some environmental stability. 

We believe that ``you don't understand it until you code it.'' We will code in R. R is free, popular in industry, and originally designed for data analysis, data visualization, modeling and estimation. We will use base R and a set of R packages that are collectively known as the ``Tidyverse.'' The Tidyverse suite is effective, popular, especially good for collaboration, well-maintaned, well-documented and easier to adopt than many alternatives. 

{\it An aside:} Students often ask whether they should learn R or Python, or why we code in R. R and Python overlap in some areas, and Python is more capable for some functions. We view R as a great supplement to Python. Python is a general-purpose toolkit and the most popular language for many data engineering and machine learning tasks. R is a specialist language designed expressly for the purposes of data analysis, data visualization and model estimation. This class will focus more on visualization and estimation, which is why we choose R as the right tool for the job. The R/Python distinction is like sportscar/truck: overlapping basic functionality, but relative performance depends on the task, such as hauling furniture or winning a race. You should learn both R and Python. You should learn other languages also, as acquiring language proficiency is an important career skill in its own right, as a successful career in analytics requires frequent re-tooling, so you can then choose the right tool for each particular environment and job. 

Class meetings will have a regular format. We will discuss key concepts, take a short break, then do a coding exercise to implement selected techniques using data. Each week's group homework will require you to revise the class script to accomplish related tasks. We will use a midterm and final exam to assess individual performance.

We seek to simulate a professional experience within the classroom. We therefore expect consistent, timely, full attendance and active participation. We will ensure sufficient time and resources to complete all deliverables. %We require no memorization.

We understand that student financial resources are limited. We rely exclusively on materials that are either free or already paid for by your tuition. We provide links to further materials for students interested in deeper learning. %We use data on a set of customers from a single product category to demonstrate complementarities between analytic techniques.

Most students will need to commit approximately 5--10 hours per week outside of class to have a successful experience. We will modify these terms and expectations as needed. Student feedback is welcome at any point. 

\vspace{0.5cm}

%\section*{Topics}
% --------------------------------------------------

%\medskip
%\hrule
%\medskip


%Here is our weekly course outline: 

%\begin{enumerate}[noitemsep]
%    \item Course Introduction \& R
%    \item Customer Data \& Data Visualization
%    \item Market Segmentation
%    \item Market Mapping \& Conjoint Analysis
%    \item Demand Estimation
%    \item Heterogeneous Demand Estimation
%    \item Price Optimization
%    \item Branding
%    \item Customer Revenue
%    \item Customer-based Corporate Valuation
%    \item Final Exam 
%\end{enumerate}


\section*{Deliverables and Grading}
% --------------------------------------------------

\medskip
\hrule
\medskip

Grades are calculated as :

\begin{itemize}[noitemsep]
    \item attendance \& intermissions (10\%), with early opt-out available
    \item homeworks (30\%)
    \item midterm (20\%)
    \item final exam (40\%)
\end{itemize}
\medskip
\textbf{Attendance \& Intermissions}: We expect full class attendance and participation. We did not require attendance in an earlier iteration, but then observed that live class attendance correlated with final grade averages at 0.46, suggesting that attending class may facilitate learning. We will assess ``Intermission'' deliverables turned in during class meetings, and we will assess in-person attendance on a few unannounced occasions. The maximum Attendance/Intermissions grade will be 90\%, so each student can miss a class meeting without excuse or consequence.

\textbf{Early attendance opt-out}: Attendance is a default requirement, but we recognize attendance may be more challenging for some students than others, given large differences in individual situations. You may opt out of the attendance grade via an optional Canvas assignment, due prior to the second class meeting. If you opt out, you can consume the class recordings asynchronously and submit individual homework deliverables on Canvas. Then the attendance portion of your grade will be dropped, and your final grade calculated using the remaining fixed proportions given above. We will video-record class meetings and try to post recording links within 1-2 days after class meetings, so students who miss class can still follow the material.\footnote{Please note, it is possible that technical problems could interfere with that process. We will do our best but cannot make any guarantees.} Students who opt out of attendance are indicating that they will be individually responsible for their learning; therefore, they will not be assigned to study groups or allowed to work with others on group assignments. Students who opt out may still attend the class in person if they like. 

Please note, we offer attendance opt-out as an individual solution for unusual hardships. We do not encourage opting out as it likely decreases learning. 

\textbf{Homeworks:} We will provide weekly assignments asking you to modify class scripts, and consider the resulting insights learned about the product, customers and market. You will submit your R scripts and answers on Canvas before the next class. Answers and script output will need to match for homework answers to count as correct. Please comment your code carefully and use the print() command to make answers clear within scripts. After week 2, homeworks can be collaborative efforts within study groups of attending students. We will assign study groups randomly in week 2 among those who did not opt out of attendance.

\textbf{Midterm and Final Exam:} Questions will test your understanding, apply key ideas, and interpret familiar code. Both exams will be on paper. You may bring any printed material you like, but device usage will be prohibited. They will not ask you to produce original code. Readings, class material and homework material will all be covered. We design the exams to take 1/3 of the allotted time, so students with extra time allowances should have enough time to finish if they take it in the regular session. 

The midterm will be during class in week 5. The final exam will be in-person at the university-appointed time and place listed on the UCSD Schedule of Classes. 

\textbf{Final Exam Alternative}: We will offer an individual student project as a final exam alternative. This is for anyone who wants to do extra work, or who plans to leave town before the exam. The project needs to use customer data in some original, useful way. The project should be proposed by the student, include a clearly delineated deliverable, and be explainably useful for the student's current interests or future career path. The proposal should be made in a private message on canvas, is subject to instructor/TA revision, and needs to be finalized by the end of week 7 at the latest. The student will need to turn in the data, analysis code, and a concise write-up. The write-up could be 1-3 pages, with a main focus on the usefulness of the project results, and a respect for the reader's time. Individual projects will be judged on a curve relative to other individual projects. Successful projects will reflect significantly more effort than the exam they replace.

\textbf{Grade calculations, Curve and Bonus}: There is no absolute grading scale. The grade scores will be standardized within each category and then averaged using the weights given above. The average grade will be curved to a B/B+, depending on the section's performance. If at least 67\% of the course evaluations are completed, each student's lowest homework score will be dropped, increasing the mean grade. Earning an A is difficult and unusual in this course due to strong competition.

\textbf{Contribution adjustments}: Some students' final grades will be adjusted based on class contributions, up to one full letter grade. Positive contributions include helping to move the class discussion forward, kindly pointing out instructor mistakes, or asking questions when something is unclear. Negative contributions include multitasking, distracting others, side conversations, tardiness, leaving early, nonconformance to classroom norms or other distractions from class discussions. If you want to maximize your positive contribution adjustment probability, we recommend you to sit in the first row, use a name tent with your preferred first name printed in dark letters, proactive participation during class discussions, and responding when called upon.

\vspace{.5cm}

\section*{Course Policies}
% --------------------------------------------------

\medskip
\hrule
\medskip

\textbf{Late Enrollment}: Students who add the course after week 1 are individually responsible for immediately catching up on all class content and deliverables. 

\textbf{Study groups and collaboration}: We will randomly assign 5-person study groups during our week 2 class meeting, among students who do not opt out from attendance. %, and each study group will receive its own dataset to analyze
Students remain individually responsible for their own homework answers and R scripts. Sharing R scripts within study groups is OK. Sharing R scripts between study groups will be considered a violation of academic integrity policies. Please be advised that we can detect this and will report it, so be very careful whom you share your scripts with, as you cannot observe or control their future distribution. There will not be any collaboration available on the final exam. Therefore, we recommend each student complete assignments individually, then meet regularly with their study group to compare results and discuss, prior to submitting R scripts on Canvas.

\textbf{Late Submissions}: Late deliverables will only be accepted in grave circumstances with documentation, such as a doctor's note for serious illness, with notification required prior to the deadline. Briefly notify us with a private Piazza message, then focus on taking care of yourself and your family. Later, provide appropriate documentation and we will agree on a reasonable extension that accommodates your situation. 

\textbf{Generative Models} We recommend and explicitly allow any use of generative models like ChatGPT or other services. We caution that you are responsible for accuracy, content and conciseness of all work, so please use all automated services with caution. We view LLMs as good sources of ideas and pointers, but unreliable sources of accurate information and poor substitutes for human understanding.


\textbf{Recommendations}: Recommendation letters depend on class performance:
\begin{itemize}
\item A : Strong letter for any analytics-related Master's or Ph.D. program in business, engineering or social science
\item A- : Letter for Master's programs may be possible depending on capacity. 
\item B+ or below : Kindly request from someone in a position to write a stronger letter.
\end{itemize}
Any letter requires a 1:1 meeting to understand your goals and motivation. Letter content will focus on the student's performance relative to the cohort, and why that is meaningful for the program in question. Professor is not qualified to write recommendations for graduate programs outside of business, engineering or social sciences. Please do not request before October of the relevant application cycle.

\textbf{Re-grade Requests}: Any request for regrading must be made in writing on Piazza within two weeks of a deliverable being assessed, or the end of the quarter, whichever comes first. The professor and/or TA will entirely regrade any such deliverable, meaning that the resulting grade change may be positive or negative, depending on the specifics of the situation. 

\bigskip

% --------------------------------------------------
\section*{Important UCSD Topics}

\medskip
\hrule
\medskip

\vspace{0.5cm}


We adopt the following policies based on university guidance.

% ----------
\subsection*{Academic Integrity}

Academic Integrity is expected of everyone at UC San Diego. This means that you must be honest, fair, responsible, respectful, and trustworthy in all of your words and actions. Lying, cheating, or any other forms of dishonesty will not be tolerated because they undermine learning and the University's ability to certify students' knowledge and abilities. Thus, any attempt to get, or help another get, a grade by cheating, lying, or dishonesty will be reported to the Academic Integrity Office and may result in sanctions. Sanctions can include a failing grade in this class and suspension or dismissal from the University. 

Integrity of scholarship is essential for an academic community. As members of the Rady School, we pledge ourselves to uphold the highest ethical standards. The University expects that both faculty and students will honor this principle and in so doing protect the validity of University intellectual work. For students, this means that all academic work will be done by the individual to whom it is assigned, without unauthorized aid of any kind.

You can learn more about academic integrity at: \newline 
\url{https://academicintegrity.ucsd.edu/}

The complete UCSD Policy on Integrity of Scholarship can be viewed at: \newline
\url{http://senate.ucsd.edu/Operating-Procedures/Senate-Manual/Appendices/2}

All aspects of the UCSD honor code apply in this course. If you are ever unsure how they apply, please ask your classmates, TA, or professor for clarification. It is much better to be conservative about honor code violations than to take a risk. %You can be suspended or expelled for cheating.

We will use automated means to detect plagiarism of submitted R scripts after week 10. Please do not share any R script outside of your own study group, as you cannot control what someone else may do with it.

\vspace{0.5cm}


% ----------
\subsection*{Students with Disabilities}

A student who has a disability or special needs and requires an accommodation in order to have equal access to the classroom must register with the Office for Students with Disabilities (OSD). The OSD will determine what accommodations may be made and provide the necessary documentation to present to the instructor and OSD liaison. 

Students requesting accommodations for this course due to a disability must provide a current Authorization for Accommodation (AFA) letter (paper or electronic) issued by the OSD. Students are required to discuss accommodation arrangements with instructors and OSD liaisons in the department 72 business hours in advance of any exams or assignments. No accommodations can be implemented retroactively.

Please visit the OSD website \url{https://osd.ucsd.edu/portal/tutorial.html} for further information or contact the Office for Students with Disabilities by phone at 858-534-4382 or via email at \href{mailto:osd@ucsd.edu}{osd@ucsd.edu}. 

\vspace{0.5cm}


% ----------
\subsection*{Non-Discrimination Policy Statement}

The University of California, in accordance with applicable Federal and State law and University policy, does not discriminate on the basis of race, color, national origin, religion, sex, gender identity, pregnancy, physical or mental disability, medical condition (cancer-related or genetic characteristics), ancestry, marital status, age, sexual orientation, citizenship, or service in the uniformed services. The University also prohibits sexual harassment. This nondiscrimination policy covers admission, access, and treatment in University programs and activities.

\vspace{0.5cm}


% ----------
\subsection*{Title IX}

The Office for the Prevention of Harassment \& Discrimination (OPHD) provides assistance to students, faculty, and staff regarding reports of bias, harassment, and discrimination. OPHD is the UC San Diego Title IX office. Title IX of the Education Amendments of 1972 is the federal law that prohibits sex discrimination in educational institutions that are recipients of federal funds. Rady students have the right to an educational environment that is free from harassment and discrimination.
 
You can make a complaint of harassment or discrimination -- or simply make an appointment to find out more information -- by contacting OPHD:

\begin{itemize}
    \item by phone at 858-534-8298
    \item by email at \href{mailto:ophd@ucsd.edu}{ophd@ucsd.edu}
    \item or online at the \href{http://ophd.ucsd.edu/policies-procedures/polpro_student.html}{Overview for Students webpage}
\end{itemize}

Students may feel more comfortable discussing their particular concern with a trusted employee. This may be a Rady student affairs staff member, a department Chair, a faculty member, or other University official. These individuals have an obligation to report incidents of sexual violence and sexual harassment to OPHD. This does not necessarily mean that a formal complaint will be filed. 
 
If you find yourself in an uncomfortable situation, ask for help. The Rady School of Management is committed to upholding University policies regarding nondiscrimination, sexual violence, and sexual harassment.

\vspace{0.5cm}


% ----------
\subsection*{Health and Well-Being}

Throughout your time at UC San Diego, you may experience a range of issues that can negatively impact your learning. These may include physical illness, housing or food insecurity, strained relationships, loss of motivation, depression, anxiety, high levels of stress, alcohol and drug problems, feeling down, interpersonal or sexual violence, or grief.

These concerns or stressful events may lead to diminished academic performance and affect your ability to participate in day-to-day activities. If there are issues related to coursework that are a source of particular stress or challenge, please speak with your professors so that we are able to support you. In addition, UC San Diego provides a number of resources to all enrolled students, including:

\begin{itemize}
    \item Counseling and Psychological Services: 858-534-3755 or \href{http://caps.ucsd.edu}{caps.ucsd.edu}
    \item Student Health Services: 858-534-3300 or \href{http://studenthealth.ucsd.edu}{studenthealth.ucsd.edu}
    \item CARE at the Sexual Assault Resource Center: 858-534-5793 or \href{http://care.ucsd.edu}{care.ucsd.edu}
    \item The Hub Basic Needs Center: 858-246-2632 or \href{http://basicneeds.ucsd.edu}{basicneeds.ucsd.edu}
\end{itemize}






\end{document}
